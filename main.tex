\documentclass{amsart}
\usepackage{tikz-cd}
\usepackage[bookmarks=true, linktocpage=true,
bookmarksnumbered=true, breaklinks=true,
pdfstartview=FitH, hyperfigures=false,
plainpages=false, naturalnames=true,
colorlinks=true, pagebackref=true,
pdfpagelabels]{hyperref}
\hypersetup{
	colorlinks,
	citecolor=blue,
	filecolor=blue,
	linkcolor=blue,
	urlcolor=blue
}

\renewcommand{\S}{\mathrm S}
\newcommand{\comch}{\texttt{ComCH}}
\newcommand{\id}{\mathrm{id}}
\newcommand{\X}{\mathcal X}

\begin{document}
\title[A C.A.S. for the study of commutativity up-to-coherent-homotopies]{A computer algebra system for the study of commutativity up-to-coherent-homotopies}
\author{Anibal M. Medina-Mardones}
\address{Max Plank Institute for Mathematics, Bonn, Germany}
\address{Department of Mathematics, University of Notre Dame, Notre Dame, IN, USA}
\email{amedinam@nd.edu}
\thanks{A.M-M. acknowledges financial support from Innosuisse grant \mbox{32875.1 IP-ICT - 1}, and the hospitality of the Max Plank Institute for Mathematics.}
\keywords{Computer algebra system, Python, homotopical algebra, operads, cohomology operations, cup product}

\begin{abstract}
	The Python package \comch\, is a lightweight specialized computer algebra system. It provides models for well known objects, the surjection and Barratt-Eccles operads, parameterizing the product structure of (co)algebras that are (co)commutative in a derived sense. The primary example treated by \comch\, of these (co)algebras is the (co)chain complex of a space.
\end{abstract} 

\maketitle

\section{Introduction}

All the basic notions of number, from the integers to the complex, are equipped with a commutative product, and it was believed until Hamilton's introduction of the quaternions, that the product of any number system must be commutative. Hamilton's discovery allowed for the consideration of other algebraic structures where commutativity was not demanded, and the conceptual shift that followed is only comparable to the effect non-euclidean geometries had in the study of shapes. Around a century later, after the development of the novel fields of topology and homotopy, mathematicians returned to the question of commutativity and identified additional levels of this property enriching the basic dichotomy. These structures correspond to coherent systems correcting homotopically the lack of strict commutativity, and constitute the focus of extensive current research in theoretical and applied topology. We mention knot theory, TQFT's, persistence homology and motion planning as some examples.

After the pioneering work of Steenrod \cite{Steenrod47, Steenrod62}, Adem \cite{Adem52}, Serre \cite{Serre53}, Cartan \cite{Cartan55}, Stashef \cite{Stasheff63}, Boardman-Vogt \cite{BoardmanVogt73}, May \cite{May70algebraic, May72geometry}, Araki-Kudo \cite{ArakiKudo56}, Dyer-Lashof \cite{DyerLashof62} and many others, today there is a rich theory of commutativity up-to-coherent-homotopies whose modern framework is provided by operads and PROPs, and where $E_n$-operads play a central role parameterizing the different levels of homotopical commutativity. In \comch, we focus on the category of differential graded modules, and consider two models of the $E_\infty$-operad equipped with filtrations by $E_n$-operads. These are respectively due to McClure-Smith \cite{McClureSmith03} and Berger-Fresse \cite{BergerFresse04} and are known as the surjection and Barratt-Eccles operads.

The homology of algebras over $E_n$-operads are not only equipped with an induced commutative product but also, when the coefficient ring is the field $\mathbb F_p = \mathbb Z/ p\mathbb Z$, with homology operations. The study of these operations at the chain level has become an important issue in topological data analysis \cite{medina2018persistence}, condensed matter physics \cite{kapustin2017fermionic}, category theory \cite{medina2020globular} and others areas. To provide researchers with effective tools for their study, \comch\, implements the constructions of \cite{medina2020chain}, and makes available for the first time chain level representations of these invariants for spaces presented simplicially or cubically.

\section*{Acknowledgment}
We gratefully acknowledge contributions from Djian Post, Wojciech Reise and Michelle Smith, stimulating conversations with Dennis Sullivan, Kathryn Hess, Greg Brumfiel, John Morgan, Ralph Kaufmann, Paolo Salvatore, Umberto Lupo, and Guillaume Tauzin, and the support of the Laboratory for Topology and Neuroscience at EPFL.

\section{Overview of \comch}

In this section we describe the overall structure and main functionalities of \comch, refering to its documentation\footnote{Currently hosted at \url{https://comch.readthedocs.io/en/latest/}} for a complete description of all its classes and their methods.

\subsection{Free modules and symmetric groups}

Let $R$ be the ring of integers or one of its quotients. In \comch\, the class \texttt{FreeModuleElement} serves to model elements in free $R$-modules, where $R$ is specified by the attribute \texttt{torsion}. Let $\S_r$ be the set of self-bijection of $\{1, \dots, r\}$ regarded as a group by composition. An element $\sigma \in \S_r$ will be represented by the sequence of its values $(\sigma(1), \dots, \sigma(r))$ and it is modeled in \comch\, using the class \texttt{SymmetricGroupElement}.

\subsection{Operads}

Operads parameterize algebraic structures on chain complexes. The precise although lengthy definition can be found for example in \cite{Markl08}. We will introduce a fundamental example from which the definition can be abstracted. Let $C$ be a chain complex of $R$-module, and consider the set $End^C(r) = Hom(C, C^{\otimes r})$ of $R$-linear maps as a chain complex in the usual way. The collection 
\begin{equation*}
End^C = \left\{End^C(r)\right\}_{r \geq 1}
\end{equation*}
is equipped with the following structure: a left action of $\S_r$ on $End^C(r)$ and composition chain maps
\begin{equation*}
\begin{tikzcd}[column sep=small, row sep=tiny]
\circ_i \colon &[-10pt] End^C(r) \otimes End^C(s) \arrow[r] & End^C(r+s-1) \\
& f \otimes g \arrow[r, |->] & (\id \otimes \cdots \otimes g \otimes \cdots \otimes \id) \circ f 
\end{tikzcd}
\end{equation*}
satisfying forms of equivariance, associativity, and unitality.

An $\mathcal O$-coalgebra structure on $C$ is a structure preserving morphism from $\mathcal O$ to $End^C$. We remark that it is also common to consider the complexes $End_A = Hom(A^{\otimes r}, A)$ assembled into an operad, with morphisms $\mathcal O \to End_A$ referred to as $\mathcal O$-algebras. The linear duality functor induces from an $\mathcal O$-coalgebra in $C$ an $\mathcal O$-algebra structure on $Hom(C, R)$. In subsection \ref{label} we will overview the examples of (co)chains of simplicial sets.

\subsection{The symmetric ring operad}

Let us consider $R[\S] = \left\{R[\S_r]\right\}_{r \geq 1}$ with $R[\S_r]$ the group ring of $\S_r$ thought of as a dg $R$-module concentrated in degree~$0$. It has the structure of an operad with left action induced from left multiplication, and compositions induced from the maps
\begin{equation} \label{eq: compostion of permutations}
\circ_i \colon \S_r \times \S_s \to \S_{r+s-1}
\end{equation}
sending a pair $(x, y)$ to the bijection $x \circ_i y$ represented diagrammatically by
\begin{equation*}
\underbrace{1 \cdots (\overbrace{i \cdots i+s-1}^y) \cdots r+s-1}_x.
\end{equation*}
More precisely, using sequences, $x \circ_i y$ is obtained by replacing the $i$-th value of $x$ with the sequence obtained by adding $i-1$ to the values of $y$, and shifting up by $s-1$ the values of $x$ greater than $s$. 

We model elements in $R[\S]$ using the class \texttt{SymmetricRingElement} which combines the classes \texttt{FreeModuleElement} and \texttt{SymmetricGroupElement}. For example, we have
\begin{verbatim}
>>> x = SymmetricRingElement({(2,3,1):-1, (1,3,2):1})
>>> y = SymmetricRingElement({(1,3,2):1, (1,2,3):2})
>>> print(x * y)
- (2,1,3) - 2(2,3,1) + (1,2,3) + 2(1,3,2)
>>> print(x.compose(y, 2))
- (2,4,3,5,1) - 2(2,3,4,5,1) + (1,5,2,4,3) + 2(1,5,2,3,4)
\end{verbatim}

\subsection{$E_\infty$-operads}

An important class of operads are those defining in each arity $r$ a resolution of the ground ring $R$ as an $R[\S_r]$-module. Such operads are called \mbox{$E_\infty$-operads}. They typically come equipped with a filtration by so called $E_n$-operads parameterizing different levels of derived commutativity, with $E_1$ corresponding to the lack of any assumed commutativity, and $E_\infty$ to the largest possible degree of homotopical commutativity. \comch\, implements models of two well known $E_\infty$-operads equipped with filtrations by $E_n$-operads. We now describe these models.

\subsection{Surjection operad}

For a positive integer $r$ let $\mathcal X(r)_d$ be the free $R$-module generated by all functions $x : \{1, \dots, d+r\} \to \{1, \dots, r\}$ modulo the $R$-submodule generated by degenerate functions, i.e., those which are either non-surjective or have a pair of equal consecutive values. There is a left action of $\mathrm S_r$ on $\mathcal X(r)$ which is up to signs defined on basis elements by $\pi \cdot x = \pi \circ x$.
We represent a surjection $x$ as the sequences of its values $\big( x(1), \dots, x(n+r) \big)$. The boundary map in this complex is defined up to signs by
\begin{equation*}
\partial x = \sum_{i = 1}^{r+d} \pm \big( x(1), \dots, \widehat{x(i)}, \dots, x(n+r) \big),
\end{equation*}
and the $i$-th composition $x \circ_i y$ of $x \in \mathcal X(r)$ and $y \in \mathcal X(s)$ by the following procedure: let $w$ be the cardinality of $x^{-1}(i)$, for every collection of \textit{ordered indices}
\begin{equation*}
1 = j_0 \leq j_1 \leq j_2 \leq \cdots \leq j_{w-1} \leq j_w = s
\end{equation*}
we construct an associated splitting of $y$
\begin{equation*}
(y(j_0), \dots, y(j_1));\ (y(j_1), \dots, y(j_2));\ \cdots \ ;\ (y(j_{w-1}), \dots, y(j_w)).
\end{equation*}
The element $x \circ_i y \in \X(r+s-1)$ is represented, up to signs, as the sum over all possible collections of order indices of the sequence obtained in the following two steps: 1) shift up by $s-1$ the values of $x$ greater than $i$, then 2) shift up by $i-1$ the values of each sequence in the associated splitting of $y$, and finally 3) replace in order the occurrences of $i$ in $x$ by the corresponding sequence in the splitting.

The elements in this operad are modeled using the class \texttt{SurjectionElement}. For example,
\begin{verbatim}
>>> x = SurjectionElement({(1,2,1,3): 1})
>>> print(x.boundary())
(2,1,3) - (1,2,3)
>>> y = SurjectionElement({(1,2,1): 1})
>>> print({x.compose(y, 1))
(1,3,1,2,1,4) - (1,2,3,2,1,4) - (1,2,1,3,1,4)
\end{verbatim}

The signs appearing in these constructions are determined by the attribute \texttt{convention} with possible values the strings \texttt{McClure-Smith} and \texttt{Berger-Fresse}, and we refer to \cite{McClureSmith03} and \cite{BergerFresse04} for their definitions.

We will now review the definition of the complexity of a surjection element. The importance of this concept is that the set of surjections elements with complexity less than $n$ defines an $E_n$-suboperad of $\mathcal X$.

The complexity of a finite binary sequence (i.e. a sequence of two distinct values) is defined as the number of consecutive distinct elements in it. For example, (1,2,2,1) and (1,1,1,2) have complexities 2 and 1 respectively. The complexity of a basis surjection element is defined as the maximum value of the complexities of its binary subsequences. Notice that for elements in $\mathcal X(2)$, complexity and degree agree. The class \texttt{SurjectionElement} models this concept with the attribute \texttt{complexity}.

\begin{verbatim}
>>> x = SurjectionElement({(1,2,1,3,1): 1})
>>> print(x.complexity)
1
\end{verbatim}

\subsection{Barratt-Eccles operad}

For a non-negative integer $r$ define the simplicial set $E(\mathrm S_r)$ by
\begin{align*}
E(\mathrm S_r)_n &= \{ (\sigma_0, \dots, \sigma_n)\ |\ \sigma_i \in \mathrm{S}_r\}, \\
d_i(\sigma_0, \dots, \sigma_n) &= (\sigma_0, \dots, \widehat{\sigma}_i, \dots, \sigma_n), \\
s_i(\sigma_0, \dots, \sigma_n) &= (\sigma_0, \dots, \sigma_i, \sigma_i, \dots, \sigma_n).
\end{align*}
It is equipped with a left $\mathrm S_r$-action defined on basis elements by
\begin{equation*}
\sigma (\sigma_0, \dots, \sigma_n) = (\sigma \sigma_0, \dots, \sigma \sigma_n).
\end{equation*}
The chain complex resulting from applying the functor of normalized $R$-chains to it is denoted $\mathcal E(r)$ and the Barratt-Eccles operad is the collection $\mathcal E = \{\mathcal E(r)\}_{r\geq0}$. To define its composition structure we use the Eilenberg-Zilber map. Let us notice that at the level of the simplicial sets $E(\S_r)$ we have compositions
\begin{equation*}
{\circ}_{i}: E(r) \times E(s) \to E(r + s - 1)
\end{equation*}
induced coordinate-wise from $\eqref{eq: compostion of permutations}$.
We define the composition of $\mathcal E$ by precomposing
\begin{equation*}
N_\bullet(\circ_i) \colon N_\bullet(E(r) \times E(s))
\longrightarrow
N_\bullet(E(r + s - 1)) = \mathcal E(r+s-1)
\end{equation*}
with the iterated Eilenberg-Zilber map
\begin{equation*}
\mathcal E(r) \otimes \mathcal E(s) =
N_\bullet(E(r)) \otimes N_\bullet(E(s))
\longrightarrow
N_\bullet(E(r) \times E(s)).
\end{equation*}
For example,
\begin{verbatim}
>>> x = BarrattEcclesElement({((1,2),(2,1)):1, ((2,1),(1,2)):2})
>>> print(x.boundary())
((1,2),) - ((2,1),)
>>> y = BarrattEcclesElement({((2,1,3),):3})
>>> print(x.compose(y, 2))
3((1,3,2,4),(3,2,4,1)) + 6((3,2,4,1),(1,3,2,4))
\end{verbatim}

We will now review the definition of the complexity of a Barratt-Eccles element. The importance of this concept is that the subset of elements with complexity less than $n$ defines an $E_n$-suboperad of $\mathcal E$.

The complexity of a finite binary sequence of elements in $\Sigma_2$ is defined as the number of consecutive distinct elements in it. For example, $((12),(21),(21),(12))$ and $((12),(12),(12),(21))$ have complexities 2 and 1 respectively. For any basis Barratt-Eccles element, and any pair of positive integers $i < j$ less than its arity, we can form a sequence as above by precomposing each permutation by the order-preserving inclusion sending $1$ and $2$ respectively to $i$ and $j$. The complexity of a basis Barratt-Eccles element is defined as the maximum over $i < j$ of the complexities of these.

\begin{verbatim}
>>> x = BarrattEcclesElement({((1,2,3), (1,3,2), (1,2,3)):1})
>>> print(x.complexity)
1
\end{verbatim}

\bibliographystyle{ieeetr}
\bibliography{bibliography}

\end{document}